%\documentclass[11pt]{amsproc}
%\documentclass[11pt]{article}
\documentclass[11pt]{article}
%\usepackage{setspace}
%\usepackage{fancyhdr}
\usepackage{fullpage}
\usepackage{graphicx}
\usepackage{amssymb}
%\usepackage{accents}
\usepackage{amsfonts}
\usepackage{amsthm}
\usepackage{amsmath}
\usepackage{eucal}
\usepackage{xypic}
\usepackage{pdfsync}
\usepackage{hyperref}
\usepackage{enumerate}



%%\setrightmargin{1in}
%\setallmargins{1in}

% Titlerule is a FAT ruler
\newcommand{\titlerule}{\rule{\linewidth}{1.5mm}}
% For comments in the draft - work in progress
\newcommand{\betainsert}[2]{\fbox{#1}\marginnote{\textsf{#2}}}

% Notes in the margin are nicer this way. HaHa
\newcommand{\marginnote}[1]{\marginpar{\scriptsize\raggedright #1}}



\def\bd{\partial}
\def\ra{\rightarrow}
\def\lra{\longrightarrow}
\def\Z{{\mathbb Z}}
\def\N{{\mathbb N}}
\def\R{{\mathbb R}}
\def\Q{{\mathbb Q}}
\def\C{{\mathbb C}}
\def\P{{\mathbb P}}
\def\K{{\mathbb K}}
\def\w{\mathcal{W}(E)}
\def\A{\mathcal{A}}
\def\B{\mathcal{B}}
\def\M{\mathcal{M}}
\def\N{\mathcal{N}}
\def\F{{\mathbb F}}
\def\p{\partial}

\newcommand*{\longhookrightarrow}{\ensuremath{\lhook\joinrel\relbar\joinrel\rightarrow}}

\newtheorem{lem}{Lemma}
\newtheorem{prop}{Proposition}
\newtheorem{thm}{Theorem}
\newtheorem{cor}{Corollary}
\newtheorem{conj}{Conjecture}
\newtheorem{defn}{Definition}
\newtheorem{claim}{Claim}
\newtheorem{ques}{Question}
\newtheorem{rem}{Remark}

\theoremstyle{remark}
\newtheorem*{prob}{Problem}
\newtheorem{ex}{Example}
\def\T{\mathbb{T}}

\begin{document}
\begin{center}
    \begin{Large} {\bf Math 510A}\\
    \end{Large}
    Haosen Wu  / Wednesday, Oct, 2019
\end{center}
%\vspace{10mm}

\subsection*{Uniqueness}
Does that suffice to tell me if the elements are separable?

In char 0 since all finite extension are separable, every finite extension is generated by finite many elements separable over base field.

In $\F_p$ We define a measure of how the claim fail. 
Definition. if E/F is and $\alpha \in E$ is algebraic over F we set $deg_F(\alpha)=deg \miu_{\alpha,F}$ $\miu$ as minimal polynomial in $\alpha/F$

We also define separable degree as follows: $sdeg_F(\alpha)=deg \miu_{\alpha,F}$ number of distinct root in an algebraic closure . It is independent of choice about the algebraic closure

Remark: 1. pick an isomorphsim ad show the number does not change.
2. sdeg() $\leq$ deg() the equality only holds when the polynomial is separable.

\begin{Lem}
    If $f\in F[x]$ is separable, and $F\subset E $, then any function of f in E is also separable. An element in an extension of E ?? that separable over F is also separable over E
\begin{proof}
 Uee f, $\delta_{f}$ coprime relation. Then we can show it satisfy the Bezout relation.
\end{proof}
\end{Lem}

\begin{Lemma} 
let E, F be fields, then $\sigma:F\rightarrow E $ is an embedding, then a polynomial $f\in F[x]$ is separable iff $\sigma_(f) \in E[x]$ is separable.
\end{Lemma}
\\Proof use Bezout relation like what we proved in last lemma.

\subsection*{separability and extension}
Claim Suppose E/F is and algebraic extension and 


\end{document}


$a^2=b^2, a,b different we $